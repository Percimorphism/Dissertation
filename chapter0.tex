%% FRONTMATTER
\begin{frontmatter}

% generate title
\maketitle

\begin{abstract}
Classical multidimensional scaling is a widely used method in dimensionality reduction and manifold learning. 
The method takes in a dissimilarity matrix and outputs a low-dimensional configuration matrix based on a spectral decomposition. 
In this dissertation, we present three noise models and analyze the resulting configuration matrices, or embeddings. 
In particular, we show that under each of the three noise models the resulting embedding gives rise to a central limit theorem. We also provide compelling simulations and real data illustrations of these central limit theorems. This perturbation analysis represents a significant advancement over previous results regarding classical multidimensional scaling behavior under randomness.

\textcolor{red}{Now the second part is for Random Forest}


\vspace{1cm}

\noindent Primary Reader: Carey E. Priebe\\
Secondary Reader: Minh Tang

\end{abstract}

\begin{acknowledgment}

The past five years in graduate school has been nothing but pure joy and excitement for me. Much of the previous statement is only true because my advisor, Carey Priebe, without whom this journey will be impossible, let along enjoyable, so I want to thank him for his unconditional support and valuable advice at all levels. To Minh Tang and Avanti Athreya, I want to express my uttermost admiration for putting up with me and explain things over and over again to me when I am confused (which is most of the time). I also want to thank my co-authors and collaborators, including Nicolas Charon, Vince Lyzinski, Youngser Park and Joshua Vogelstein, who have all offered their help with patience in discussions over the past few years.


Special thanks to Dan Naiman, Daniel Robinson for their time and suggestions in my Candidacy Exam and Raman Arora and 
Katia Consani for their time and suggestions in my Graduate Board Exam.

I also want express my appreciation for the faculty and staff of the Applied Mathematicss and Statistics Department at Johns Hopkins University. In particular, my thanks to John Wierman,  Edward Scheinerman,  Fred Torcaso, Donniell Fishkind, Tam\'{a}s Budav\'{a}ri for their kindness advice; and to Kristin Bechtel, Sandy Kirt, and Ann Gibbins for their help.
My deepest thanks to my fellow students and friends at Hopkins for coffee, soccer games, and all the time we have spent together.
Finally, my warmest thanks to my parents, Weiyang Li and Huiping Xu, whose unconditional love made me who I am (well, at least the good part). 

\end{acknowledgment}

\begin{dedication}
 
This thesis is dedicated to my parents and the Schaufelds

\end{dedication}

% generate table of contents
\tableofcontents

% generate list of tables
\listoftables

% generate list of figures
\listoffigures

\end{frontmatter}
